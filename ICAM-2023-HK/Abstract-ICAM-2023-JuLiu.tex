%%%%%%  TALK ABSTRACT, ICAM 2020, LATEX SOURCE FILE (version 1.1) %%%%%%
\documentclass[12pt]{article}
\usepackage{amsmath}
\usepackage{amssymb}
\usepackage{amsfonts}
\begin{document}
%% PLEASE FILL THE NAME(S) OF AUTHOR(S) HERE. IF THERE ARE MORE THAN ONE AUTHOR,
%% PLEASE MARK THE SPEAKER BY PUTTING "${}^*$" RIGHT BEHIND HIS/HER NAME.
%% EXAMPLE:
%% \newcommand\nameLBJ{First Author, Second Author${}^*$,
%% Third Author}
\newcommand\AuthorNameLBJ{Ju Liu }

%% PLEASE FILL THE SPEAKER'S AFFILIATION HERE
%% EXAMPLE: \def\affiliationLBJ{Department of Mathematics, City University of Hong Kong, Hong Kong}
\newcommand\affiliationLBJ{Department of Mechanics and Aerospace Engineering, Southern University of Science and Technology, Shenzhen}

%% PLEASE FILL THE SPEAKER'S EMAIL ADDRESS HERE
\newcommand\emailLBJ{jliu36@sustech.edu.cn, liujuy@gmail.com}

%% PLEASE FILL THE TITLE OF YOUR TALK HERE
\newcommand\titleOfTalkLBJ{A revisit of a viscoelasticity theory}
%% PLEASE FILL THE ABSTRACT OF YOUR TALK HERE
\newcommand\abstractOfTalkLBJ{The viscoelasticity theory proposed in [1,2] has gained popularity over the years, largely because it is amenable to finite element implementation and convenient in accounting for material anisotropy. Recently, a finite-time blow-up solution has been identified [3], signifying an alerting issue concerning its theoretical root. The lack of a thermodynamic foundation has been viewed as a major drawback of this model.

In this talk, I will address the issue by providing a complete thermomechanical theory for the aforementioned finite viscoelasticity model [4]. The derivation elucidates the origin of the evolution equations of that model, with a few non-negligible differences. It is also shown that the conjugate variable and non-equilibrium stress should be differentiated, an issue that has been ignored in prior works. I will discuss the relaxation property of the non-equilibrium stress in the thermodynamic equilibrium limit and its implication on the form of free energy, which clarifies the failure of a classical model based on the identical polymer chain assumption.

Based on the consistent framework, a set of energy-momentum consistent schemes is constructed for finite viscoelasticity using a \textit{strain-driven} constitutive integration scheme and a generalized \textit{directionality property} for the stress-like variables. I adopt a suite of smooth generalization of the Taylor-Hood element based on Non-Uniform Rational B-Splines for spatial discretization. The element is further enhanced by the grad-div stabilization to improve the discrete mass conservation. I will also discuss recent advancements in designing non-singular algorithmic stresses for energy-momentum consistent schemes [5]. Numerical examples will be provided to justify the effectiveness of the proposed methodology.

\vspace{1cm}

\noindent [1] J.C. Simo. On a fully three-dimensional finite-strain viscoelastic damage model: formulation and computational aspects. {\em Computer Methods in Applied Mechanics and Engineering}, 60:153-173, 1987. 

\noindent [2] J.S. Simo and T.J.R. Hughes. Computational Inelasticity. Springer Science \& Business Media, 2006.

\noindent [3] S. Govindjee, T. Potter, and J. Wilkening. Dynamic stability of spinning viscoelastic cylinders at finite deformation. {\em International journal of solids and structures}, 51:3589-3603, 2014.

\noindent [4] J. Liu, M. Latorre, and A.L. Marsden. A continuum and computational framework for viscoelastodynamics: I. Finite deformation linear models. {\em Computer Methods in Applied Mechanics and Engineering}, 385:114059, 2021. 

\noindent [5] J. Liu, On the design of non-singular, energy-momentum consistent integrators for nonlinear dynamics using energy splitting and perturbation techniques. {\em Journal of Computational Physics}, accepted.
 }

%%%%%%%%%%%%%%%%%%%%%%%%%%%%%%%%%%%%%%%%%%%%%%%%%%%%%%%%%%%%%%%%%%%%%%%%%%%%%%%%
%%  THE CODE BELOW IS FOR LBJ USE ONLY. PLEASE DO NOT ALTER IT!               %%
%%  THIS ENABLES OUR COMPUTER TO PROCESS THE FILES AUTOMATICALLY IN BATCH.    %%
%%%%%%%%%%%%%%%%%%%%%%%%%%%%%%%%%%%%%%%%%%%%%%%%%%%%%%%%%%%%%%%%%%%%%%%%%%%%%%%%

\noindent\textbf{\large\ignorespaces\titleOfTalkLBJ}$\,$\\*
\noindent\textsc{\ignorespaces\AuthorNameLBJ}$\,$\\*
\noindent\ignorespaces\affiliationLBJ$\,$\\*
\noindent\textit{Email: }\texttt{\ignorespaces\emailLBJ}$\,$\\*
\noindent\rule{\textwidth}{0.15mm}$\,$\\*
 \abstractOfTalkLBJ
\par
\end{document}

%%%%%%%%%%%end of abstract%%%%%%%%%%%
